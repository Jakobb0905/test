% !TEX TS-program = pdflatex
% !TEX encoding = UTF-8 Unicode
% !TEX root = ../main.tex
% !TEX spellcheck = en-US
% ****************************************************************************************
% File: introduction.tex
% Author: Jakob Spindler
% Date: 2024-06-01
% ****************************************************************************************
\chapter{Introduction}
\label{chapter:introduction}
The \texttt{turtlesimAutomata} package is a ROS2 (Robot Operating System) \autocite{noauthor_ros_nodate} package that uses a finite state machine-like structure to control a turtle in the turtlesim \autocite{noauthor_introducing_nodate} environment.
It's purpose is to provide an entry-level example of a ROS2 package to acquaint the user with the basic concepts of ROS2.
The \texttt{turtlesimAutomata} package is closely intertwined with the \texttt{turtlesim} package, which is a simple simulator for a mobile robot in the shape of a turtle.

\section{Assignement}
\label[section]{section:assignment}
The assignement stated the following requirements for the \texttt{turtlesimAutomata} package:
\begin{itemize}
    \item The \texttt{turtlesimAutomata} package should work closely together with the \texttt{turtlesim} package.
    \item The turtle shall start off in a random direction
    \item The turtle shall move in a straight line until it reaches the edge of the turtlesim window, at which point it shall make a 90 degree turn in clockwise direction and continue moving in the new direction.
\end{itemize}





% EOF