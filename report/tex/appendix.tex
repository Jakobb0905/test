% !TEX TS-program = pdflatex
% !TEX encoding = UTF-8 Unicode
% !TEX root = ../main.tex
% !TEX spellcheck = en-US
% ****************************************************************************************
% File: appendix.tex
% Author: Jakob Spindler
% Date: 2024-06-01
% ****************************************************************************************
\chapter{\texttt{turtlesimAutomata} source code}
\label{chapter:source_code}



% Define custom colors for the listings
\definecolor{codegray}{rgb}{0.5,0.5,0.5}
\definecolor{codegreen}{rgb}{0,0.6,0}
\definecolor{codepurple}{rgb}{0.58,0,0.82}
\definecolor{backcolour}{rgb}{0.95,0.95,0.92}

% Configure the listings package for C++
\lstdefinestyle{mystyle}{
    backgroundcolor=\color{backcolour},
    commentstyle=\color{codegreen},
    keywordstyle=\color{magenta},
    numberstyle=\tiny\color{codegray},
    stringstyle=\color{codepurple},
    basicstyle=\ttfamily\footnotesize,
    breakatwhitespace=false,
    breaklines=true,
    captionpos=b,
    keepspaces=true,
    numbers=left,
    numbersep=5pt,
    showspaces=false,
    showstringspaces=false,
    showtabs=false,
    tabsize=2,
    language=C++,
    morekeywords={rclcpp, std_msgs, rcl_interfaces}, % Add more custom keywords if necessary
}

% Configure the listings package for XML
\lstdefinestyle{xmlstyle}{
    backgroundcolor=\color{backcolour},
    commentstyle=\color{codegreen},
    keywordstyle=\color{magenta},
    numberstyle=\tiny\color{codegray},
    stringstyle=\color{codepurple},
    basicstyle=\ttfamily\footnotesize,
    breakatwhitespace=false,
    breaklines=true,
    captionpos=b,
    keepspaces=true,
    numbers=left,
    numbersep=5pt,
    showspaces=false,
    showstringspaces=false,
    showtabs=false,
    tabsize=2,
    language=XML,
    morekeywords={version, xml, package, format, name, version, description, maintainer, email, license, depend, buildtool_depend, exec_depend, member_of_group, test_depend, export, build_type}, % Add more custom keywords if necessary
}

% Configure the listings package for CMake
\lstdefinestyle{cmakestyle}{
    backgroundcolor=\color{backcolour},
    commentstyle=\color{codegreen},
    keywordstyle=\color{magenta},
    numberstyle=\tiny\color{codegray},
    stringstyle=\color{codepurple},
    basicstyle=\ttfamily\footnotesize,
    breakatwhitespace=false,
    breaklines=true,
    captionpos=b,
    keepspaces=true,
    numbers=left,
    numbersep=5pt,
    showspaces=false,
    showstringspaces=false,
    showtabs=false,
    tabsize=2,
    language=make,
    morekeywords={cmake_minimum_required, project, if, endif, find_package, add_compile_options, ament_target_dependencies, install, include_directories, ament_package, set, add_executable, ament_lint_auto_find_test_dependencies, target_include_directories, target_compile_features, ament_lint_auto, BUILD_INTERFACE, INSTALL_INTERFACE, DESTINATION, DIRECTORY, comment}, % Add more custom keywords if necessary
}








\lstset{style=mystyle}


\begin{lstlisting}[caption={\texttt{/start\_turtle} source code}, label={lst:start_turtle_source_code}]
    #include "rclcpp/rclcpp.hpp"
    #include "rcl_interfaces/msg/log.hpp"
    #include "std_msgs/msg/float64.hpp"
    #include "std_msgs/msg/bool.hpp"
    #include <cstdlib> // Include the C standard library for random number generation
    
    class StartTurtle : public rclcpp::Node
    {
    public:
        StartTurtle() : Node("start_turtle")
        {
            subscription_ = this->create_subscription<rcl_interfaces::msg::Log>(
                "/rosout", 10, std::bind(&StartTurtle::log_callback, this, std::placeholders::_1));
            
            angle_publisher_ = this->create_publisher<std_msgs::msg::Float64>("/turn_angle", 1);
            //drive_publisher_ = this->create_publisher<std_msgs::msg::Bool>("/drive_turtle", 1);
    
            RCLCPP_INFO(this->get_logger(), "Started 'start_turtle' node");
        }
    
    private:
        void log_callback(const rcl_interfaces::msg::Log::SharedPtr msg)
        {
            if (msg->msg.find("Spawning turtle") != std::string::npos)
            {
                RCLCPP_INFO(this->get_logger(), "Detected 'Spawning turtle' message. Proceeding with actions.");
                std::this_thread::sleep_for(std::chrono::milliseconds(1500));
                publish_random_angle();
                //std::this_thread::sleep_for(std::chrono::milliseconds(1500));
                //publish_drive_true();
                rclcpp::shutdown();
            }
        }
    
        void publish_random_angle()
        {
            // Generate a random angle in range of 45 to 90, 135 to 180, 225 to 270, 315 to 360 degrees
            // this ensures, that the turtle can turn away from the wall with one 90 degree clockwise turn,
            // thus keeping the visuals interesting.
            // (all other angles result in the turtle just drawing a straight line between two walls,
            // as it alsways ahs to perofrm 180 degree rotations)
            /*
            double angle_degrees = rand() % 45;
            int modifier = rand() % 4;
            angle_degrees *=  -1;
            angle_degrees += modifier * 90;
            */
    
            double angle_degrees = generate_random_angle();
    
            auto angle_msg = std_msgs::msg::Float64();
            angle_msg.data = angle_degrees;
            angle_publisher_->publish(angle_msg);
    
            RCLCPP_INFO(this->get_logger(), "Published random angle: %.2f degrees", angle_msg.data);
        }
    
        void publish_drive_true()
        {
            auto drive_msg = std_msgs::msg::Bool();
            drive_msg.data = true;
            drive_publisher_->publish(drive_msg);
    
            RCLCPP_INFO(this->get_logger(), "Published 'true' to /drive_turtle");
        }
    
        double generate_random_angle()
        {
            // Seed the random number generator
            srand(time(nullptr));
    
            // Define the ranges for each quadrant
            int lower_bounds[] = {45, 135, 225, 315};
            int upper_bounds[] = {90, 180, 270, 360};
    
            // Randomly select a quadrant
            int quadrant = rand() % 4;
    
            // Generate a random angle within the selected quadrant
            int angle = rand() % (upper_bounds[quadrant] - lower_bounds[quadrant] + 1) + lower_bounds[quadrant];
    
            return angle;
        }
    
        rclcpp::Subscription<rcl_interfaces::msg::Log>::SharedPtr subscription_;
        rclcpp::Publisher<std_msgs::msg::Float64>::SharedPtr angle_publisher_;
        rclcpp::Publisher<std_msgs::msg::Bool>::SharedPtr drive_publisher_;
    };
    
    int main(int argc, char *argv[])
    {
        rclcpp::init(argc, argv);
        auto node = std::make_shared<StartTurtle>();
        rclcpp::spin(node);
        std::this_thread::sleep_for(std::chrono::seconds(5));
        rclcpp::shutdown();
        return 0;
    }
    
\end{lstlisting}


\begin{lstlisting}[caption={\texttt{/drive\_turtle\_continuously} source code}, label={lst:drive_turtle_continuously_source_code}]
    #include "rclcpp/rclcpp.hpp"
    #include "std_msgs/msg/bool.hpp"
    #include "geometry_msgs/msg/twist.hpp"
    
    class DriveTurtleContinuously : public rclcpp::Node
    {
    public:
        DriveTurtleContinuously() : Node("drive_turtle_continuously"), drive_turtle_(false)
        {
            subscription_ = this->create_subscription<std_msgs::msg::Bool>(
                "/drive_turtle", 10, std::bind(&DriveTurtleContinuously::drive_callback, this, std::placeholders::_1));
            cmd_vel_publisher_ = this->create_publisher<geometry_msgs::msg::Twist>("turtle1/cmd_vel", 10);
            timer_ = this->create_wall_timer(
                std::chrono::milliseconds(100), std::bind(&DriveTurtleContinuously::publish_velocity, this));
    
            RCLCPP_INFO(this->get_logger(), "Started 'drive_turtle_continuously' node");
        }
    
    private:
        void drive_callback(const std_msgs::msg::Bool::SharedPtr msg)
        {
            drive_turtle_ = msg->data;
            if (!drive_turtle_) {
                // Immediately stop the turtle
                auto stop_msg = geometry_msgs::msg::Twist();
                cmd_vel_publisher_->publish(stop_msg);
                RCLCPP_INFO(this->get_logger(), "Received stop command. Stopping turtle.");
            } else {
                RCLCPP_INFO(this->get_logger(), "Received drive command. Starting turtle.");
            }
        }
    
        void publish_velocity()
        {
            if (drive_turtle_)
            {
                auto twist_msg = geometry_msgs::msg::Twist();
                twist_msg.linear.x = 3.0;  // Set linear x velocity to 1.0
                cmd_vel_publisher_->publish(twist_msg);
                //RCLCPP_INFO(this->get_logger(), "Publishing cmd_vel to move turtle.");
            }
        }
    
        rclcpp::Subscription<std_msgs::msg::Bool>::SharedPtr subscription_;
        rclcpp::Publisher<geometry_msgs::msg::Twist>::SharedPtr cmd_vel_publisher_;
        rclcpp::TimerBase::SharedPtr timer_;
        bool drive_turtle_;
    };
    
    int main(int argc, char *argv[])
    {
        rclcpp::init(argc, argv);
        auto node = std::make_shared<DriveTurtleContinuously>();
        rclcpp::spin(node);
        rclcpp::shutdown();
        return 0;
    }
    
\end{lstlisting}


\begin{lstlisting}[caption={\texttt{/turn\_turtle} source code}, label={lst:turn_turtle_source_code}]
    #include "rclcpp/rclcpp.hpp"
    #include "std_msgs/msg/float64.hpp"
    #include "std_msgs/msg/bool.hpp"
    #include "turtlesim/srv/teleport_relative.hpp"
    #include "geometry_msgs/msg/twist.hpp"
    
    class TurnTurtle : public rclcpp::Node
    {
    public:
        TurnTurtle() : Node("turn_turtle")
        {
            subscription_ = this->create_subscription<std_msgs::msg::Float64>(
                "/turn_angle", 1, std::bind(&TurnTurtle::angle_callback, this, std::placeholders::_1));
            
            client_ = this->create_client<turtlesim::srv::TeleportRelative>("turtle1/teleport_relative");
            //cmd_vel_publisher_ = this->create_publisher<geometry_msgs::msg::Twist>("turtle1/cmd_vel", 1);
            drive_turtle_publisher_ = this->create_publisher<std_msgs::msg::Bool>("/drive_turtle", 1);
    
            RCLCPP_INFO(this->get_logger(), "Started 'turn_turtle' node");
        }
    
    private:
        void angle_callback(const std_msgs::msg::Float64::SharedPtr msg)
        {
            double angle_degrees = msg->data;
            RCLCPP_INFO(this->get_logger(), "Received angle: %.2f degrees. Stopping and turning turtle.", angle_degrees);
            turn_turtle(angle_degrees);
        }
    
        void turn_turtle(double angle_degrees)
        {
            // Stop the turtle's current movement
            /*
            auto stop_message = geometry_msgs::msg::Twist();
            stop_message.linear.x = 0.0;
            stop_message.angular.z = 0.0;
            cmd_vel_publisher_->publish(stop_message);
            */
            rclcpp::sleep_for(std::chrono::seconds(1)); // Wait for a second to ensure it stops
    
            if (!client_->wait_for_service(std::chrono::seconds(1))) {
                RCLCPP_ERROR(this->get_logger(), "Service not available. Make sure the turtlesim node is running.");
                return;
            }
    
            auto request = std::make_shared<turtlesim::srv::TeleportRelative::Request>();
            request->linear = 0.0;
            request->angular = angle_degrees * (M_PI / 180.0); // Convert degrees to radians
    
            auto result = client_->async_send_request(request);
            // Don't wait for the service to respond! It does not.
    
            // Notify that the turning is finished
            auto drive_message = std_msgs::msg::Bool();
            drive_message.data = true;
            rclcpp::sleep_for(std::chrono::seconds(1));
            drive_turtle_publisher_->publish(drive_message);
        }
    
        rclcpp::Subscription<std_msgs::msg::Float64>::SharedPtr subscription_;
        rclcpp::Client<turtlesim::srv::TeleportRelative>::SharedPtr client_;
        rclcpp::Publisher<geometry_msgs::msg::Twist>::SharedPtr cmd_vel_publisher_;
        rclcpp::Publisher<std_msgs::msg::Bool>::SharedPtr drive_turtle_publisher_;
    };
    
    int main(int argc, char *argv[])
    {
        rclcpp::init(argc, argv);
        auto node = std::make_shared<TurnTurtle>();
        rclcpp::spin(node);
        rclcpp::shutdown();
        return 0;
    }
    
\end{lstlisting}


\begin{lstlisting}[caption={\texttt{/edge} source code}, label={lst:edge_source_code}]
#include "rclcpp/rclcpp.hpp"
#include "rcl_interfaces/msg/log.hpp"
#include "std_msgs/msg/float64.hpp"
#include "std_msgs/msg/bool.hpp"

class Edge : public rclcpp::Node
{
public:
    Edge() : Node("edge"), should_look_for_edges_(false)
    {
        drive_turtle_subscription_ = this->create_subscription<std_msgs::msg::Bool>(
            "/drive_turtle", 1, std::bind(&Edge::drive_turtle_callback, this, std::placeholders::_1));
        turn_angle_publisher_ = this->create_publisher<std_msgs::msg::Float64>("/turn_angle", 1);
        drive_turtle_publisher_ = this->create_publisher<std_msgs::msg::Bool>("/drive_turtle", 1);

        RCLCPP_INFO(this->get_logger(), "Started 'edge' node");
    }

private:
    void subscribe_to_rosout()
    {
        rosout_subscription_ = this->create_subscription<rcl_interfaces::msg::Log>(
            "/rosout", 1, std::bind(&Edge::listener_callback, this, std::placeholders::_1));
    }

    void drive_turtle_callback(const std_msgs::msg::Bool::SharedPtr msg)
    {
        should_look_for_edges_ = msg->data;
        if (should_look_for_edges_)
        {
            RCLCPP_INFO(this->get_logger(), "Received true on /drive_turtle. Subscribing to /rosout for edge detection.");
            subscribe_to_rosout();
        }
        else
        {
            RCLCPP_INFO(this->get_logger(), "Received false on /drive_turtle. Unsubscribing from /rosout.");
            rosout_subscription_.reset();
        }
    }

    void listener_callback(const rcl_interfaces::msg::Log::SharedPtr msg)
    {
        if (should_look_for_edges_ && msg->msg.find("Oh no! I hit the wall!") != std::string::npos)
        {
            RCLCPP_INFO(this->get_logger(), "Received message: 'Oh no! I hit the wall!'. Publishing angle and stopping turtle...");
            publish_drive_turtle(false);
            publish_turn_angle();
            // Unsubscribing from /rosout for 5 milliseconds to avoid multiple wall-hitting messages
            /**/
            rosout_subscription_.reset();
            std::this_thread::sleep_for(std::chrono::milliseconds(50));
            //subscribe_to_rosout();
        }
    }

    void publish_turn_angle()
    {
        auto message = std_msgs::msg::Float64();
        message.data = -90.0; // 90 degrees clockwise
        turn_angle_publisher_->publish(message);
    }

    void publish_drive_turtle(bool data)
    {
        auto message = std_msgs::msg::Bool();
        message.data = data;
        drive_turtle_publisher_->publish(message);
    }

    rclcpp::Subscription<rcl_interfaces::msg::Log>::SharedPtr rosout_subscription_;
    rclcpp::Subscription<std_msgs::msg::Bool>::SharedPtr drive_turtle_subscription_;
    rclcpp::Publisher<std_msgs::msg::Float64>::SharedPtr turn_angle_publisher_;
    rclcpp::Publisher<std_msgs::msg::Bool>::SharedPtr drive_turtle_publisher_;
    bool should_look_for_edges_;
};

int main(int argc, char *argv[])
{
    rclcpp::init(argc, argv);
    auto node = std::make_shared<Edge>();
    rclcpp::spin(node);
    rclcpp::shutdown();
    return 0;
}
\end{lstlisting}



\chapter{Auxiliary files}
\label{chapter:aux_files}


\lstset{style=cmakestyle}
\begin{lstlisting}[caption={CMakeLists.txt}, label={lst:CMakeLists}]

    cmake_minimum_required(VERSION 3.8)
    project(turtlesimAutomata)
    
    if(CMAKE_COMPILER_IS_GNUCXX OR CMAKE_CXX_COMPILER_ID MATCHES "Clang")
      add_compile_options(-Wall -Wextra -Wpedantic)
    endif()
    
    # find dependencies
    find_package(ament_cmake REQUIRED)
    # uncomment the following section in order to fill in
    # further dependencies manually.
    # find_package(<dependency> REQUIRED)
    find_package(rclcpp REQUIRED)
    find_package(std_msgs REQUIRED)
    #find_package(rosidl_default_generators REQUIRED)
    #find_package(turtlesimAutomata REQUIRED)
    find_package(turtlesim REQUIRED)
    find_package(geometry_msgs REQUIRED)
    find_package(rcl_interfaces)
    
    
    #rosidl_generate_interfaces(${PROJECT_NAME}
    #  "srv/acknowledge.srv"
    #  DEPENDENCIES std_msgs
    #)
    
    # Include directories
    include_directories(include)
    
    
    
    add_executable(edge src/edge.cpp)
    ament_target_dependencies(edge rclcpp std_msgs)
    #  $<BUILD_INTERFACE:${CMAKE_CURRENT_SOURCE_DIR}/include>
    #  $<INSTALL_INTERFACE:include>)
    #target_compile_features(edge PUBLIC c_std_99 cxx_std_17)  # Require C99 and C++17
    
    add_executable(turn_turtle src/turn_turtle.cpp)
    ament_target_dependencies(turn_turtle rclcpp std_msgs turtlesim geometry_msgs)
    
    add_executable(drive_turtle_continuously src/drive_turtle_continuously.cpp)
    ament_target_dependencies(drive_turtle_continuously std_msgs rclcpp geometry_msgs)
    
    add_executable(start_turtle src/start_turtle.cpp)
    ament_target_dependencies(start_turtle std_msgs rclcpp rcl_interfaces)
    target_include_directories(turn_turtle PUBLIC
      $<BUILD_INTERFACE:${CMAKE_CURRENT_SOURCE_DIR}/include>  
      $<INSTALL_INTERFACE:include>)
    target_compile_features(turn_turtle PUBLIC c_std_99 cxx_std_17)  # Require C99 and C++17
    
    
    install(TARGETS
      edge
      turn_turtle
      drive_turtle_continuously
      start_turtle
      
      DESTINATION lib/${PROJECT_NAME})
    
      
    #this is important for the launchfile to be accesible from the ros2_ws
    install(DIRECTORY launch
    DESTINATION share/${PROJECT_NAME})
    
    
    
    if(BUILD_TESTING)
      find_package(ament_lint_auto REQUIRED)
      # the following line skips the linter which checks for copyrights
      # comment the line when a copyright and license is added to all source files
      set(ament_cmake_copyright_FOUND TRUE)
      # the following line skips cpplint (only works in a git repo)
      # comment the line when this package is in a git repo and when
      # a copyright and license is added to all source files
      set(ament_cmake_cpplint_FOUND TRUE)
      ament_lint_auto_find_test_dependencies()
    endif()
    
    ament_package()
    

\end{lstlisting}



\lstset{style=xmlstyle}
\begin{lstlisting}[caption={Package.xml}, label={lst:packagexml}]
    <?xml version="1.0"?>
<?xml-model href="http://download.ros.org/schema/package_format3.xsd" schematypens="http://www.w3.org/2001/XMLSchema"?>
<package format="3">
  <name>turtlesimAutomata</name>
  <version>0.0.0</version>
  <description>TODO: Package description</description>
  <maintainer email="jakobspindler@gmx.at">jakob</maintainer>
  <license>Apache-2.0</license>

  <!--depend>std_msgs</depend>
  <buildtool_depend>rosidl_default_generators</buildtool_depend>
  <exec_depend>rosidl_default_runtime</exec_depend>
  <member_of_group>rosidl_interface_packages</member_of_group-->

  <buildtool_depend>ament_cmake</buildtool_depend>

  <depend>rclcpp</depend>
  <depend>std_msgs</depend>
  <depend>turtlesim</depend>
  <depend>geometry_msgs</depend>
  <depend>rcl_interfaces</depend>



  <test_depend>ament_lint_auto</test_depend>
  <test_depend>ament_lint_common</test_depend>

  <exec_depend>ros2launch</exec_depend>

  <export>
    <build_type>ament_cmake</build_type>
  </export>
</package>

\end{lstlisting}

\lstset{style=xmlstyle}

\begin{lstlisting}[caption={turtlesimAutomata\_launch.xml}, label={lst:launchfile}]
    <launch>
    <!-- Launch the drive_turtle_continuously node -->
    <node pkg="turtlesimAutomata" exec="drive_turtle_continuously" name="drive_turtle_continuously" output="screen"/>
    
    <!-- Launch the edge node -->
    <node pkg="turtlesimAutomata" exec="edge" name="edge" output="screen"/>
    
    <!-- Launch the turn_turtle node -->
    <node pkg="turtlesimAutomata" exec="turn_turtle" name="turn_turtle" output="screen"/>
  
    <!-- Launch the start_turtle node -->
    <node pkg="turtlesimAutomata" exec="start_turtle" name="start_turtle" output="screen"/>
  
    <!-- Launch the turtlesim node -->
    <node pkg="turtlesim" exec="turtlesim_node" name="turtlesim_node" output="screen"/>
  </launch>
\end{lstlisting}


%EOF