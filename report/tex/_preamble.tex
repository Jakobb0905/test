% !TEX TS-program = pdflatex
% !TEX encoding = UTF-8 Unicode
% !TEX root = ../main.tex
% !TEX spellcheck = en-US
% ****************************************************************************************
% File: _preamble.tex
% Author: Patrick Haselwanter
% Date: 2023-10-28
% ****************************************************************************************

% ****************************************************************************************
% General settings (input encoding, font encoding, font, language)
% ****************************************************************************************
\usepackage[utf8]{inputenc} % character encoding used in input file
\usepackage[T1]{fontenc} % specifies the encoding used in the fonts
\usepackage{lmodern} % provides more support for non-ASCII characters than cm-super
\usepackage{microtype} % improves line-filling when using PDFLaTeX
\usepackage[ngerman,english]{babel} %  last language is considered the main one
\renewcommand{\familydefault}{\sfdefault} % select a sans serif font family 

% \pdfsuppresswarningpagegroup=1 % suppress warning when including PDFs with page groups

% ****************************************************************************************
% Basic macros for thesis
% ****************************************************************************************
\newcommand{\authorAName}{Jakob Spindler}
\newcommand{\authorAContact}{sj0458@mci4me.at}

\newcommand{\courseName}{Mobile Robotics}
\newcommand{\courseCode}{MECH-M-2-ROB}
\newcommand{\department}{Department of Technology \& Life Sciences}
\newcommand{\docTitle}{TurtlesimAutomata - dipping the toe into ROS}
\newcommand{\docType}{Software report}
\newcommand{\studyProgram}{Master's program Mechatronics \& Smart Technologies}
\newcommand{\studyYear}{MA-MECH-23-VZ}
\newcommand{\lecturerName}{Daniel McGuiness}
\newcommand{\lecturerContact}{}
\newcommand{\university}{Management Center Innsbruck}

% ****************************************************************************************
% Drawing and plotting, scientific packages, 
% ****************************************************************************************
% For use of subfigure environment
\usepackage{subcaption}

% For use of cmidrule in table environment
\usepackage{booktabs}

\usepackage{makecell}

% Handling of images
\usepackage{graphicx}
\graphicspath{{./img/}}

% Colour support
\usepackage[table]{xcolor}

% Handling of MATLAB code
%\usepackage{mcode}

% Tabulars with adjustable-width columns
\usepackage{tabularx}
\usepackage{multirow}

% Sketching and importing MATLAB plots
\usepackage{pgfplots}
\usepackage{grffile}
\pgfplotsset{compat=newest}
\usetikzlibrary{plotmarks}
\usetikzlibrary{arrows.meta}
\usetikzlibrary{patterns}
\usepgfplotslibrary{patchplots}
\pgfplotsset{plot coordinates/math parser=false}
\newlength\figureheight
\newlength\figurewidth

% Typesetting electrical networks
\usepackage{circuitikz}

%svg files
\usepackage{svg}

% scientific packages
\usepackage{amsmath}
\usepackage{amsfonts}
\usepackage{xfrac}
\usepackage{siunitx}
\AtBeginDocument{\sisetup{
		mode=match,
		unit-font-command = \mathrm,
		reset-text-family=false,
		reset-text-series=false,
		reset-text-shape=false,
		exponent-product=\cdot
	}}
\DeclareSIUnit\unity{1} % can be used for dimensionless quantities
\DeclareSIUnit\sample{Sa}
% ****************************************************************************************
% Referencing and citing
% ****************************************************************************************
% Caption settings
\usepackage[
	format=plain, % typeset as normal paragraph
	labelformat=simple, % typeset label as name and number
	labelsep=period, % caption label and text separated by period and space
	textformat=simple, % caption text typeset as is
	justification=justified, % typset caption as normal paragraph
	singlelinecheck=true, % automatically center short captions
	font=small,
	labelfont=bf, % set bold font for label
	width=.75\textwidth % set fixed width for caption
]{caption}
\captionsetup[table]{position=top}
\captionsetup[figure]{position=bottom}

% Hypertext marks (should be loaded last but before geometry)
\usepackage[hyperindex]{hyperref}
% Extension options
\hypersetup{
	colorlinks, % colours the text of links and anchors (instead of borders)
	linkcolor={blue!65!black},
	citecolor={blue!65!black},
	urlcolor={blue!65!black}
}
% PDF display and information options
\hypersetup{
	pdftitle={\docTitle},
	pdfsubject={\docType},
	pdfauthor={\authorAName},
	pdfkeywords={},
	pdfcreator={pdflatex},
	pdfproducer={LaTeX with hyperref}
}

% formatting of cross-references
\usepackage[capitalise]{cleveref}
\crefformat{equation}{(#2#1#3)}
\Crefformat{equation}{Equation~(#2#1#3)}

%matlab prettifier
\usepackage{matlab-prettifier}

%C++ 
\usepackage{listings}
\usepackage{xcolor}

% ****************************************************************************************
% Bibliography settings
% ****************************************************************************************
% template from https://www.ieee.org/conferences/publishing/templates.html
\usepackage[
	backend=biber,
	style=numeric,
	natbib=true,
	sorting=nty
]{biblatex}
%\addbibresource{C:/Users/pober/OneDrive/StudiumMaster/latex/Master.bib}
%\addbibresource{Master2.bib}
% \addbibresource{../../../../bib/mybib.bib}

% \addbibresource{C:\\Users\\pober\\OneDrive\\StudiumMaster\\latex\\Master.bib} % point BibTEX at the .bib files
% \bibliographystyle{IEEEtran} % choose the reference style
% ****************************************************************************************
% Glossary (acronyms, list of symbols) settings
% ****************************************************************************************
% \usepackage[acronym,nomain,nonumberlist,nopostdot,sort=def,toc]{glossaries}
% \renewcommand*{\glstextformat}[1]{\textcolor{black}{#1}} % make links appear black
% \newglossary[slg]{symbolslist}{syi}{syg}{List of Symbols} % define custom glossary
% \glsaddkey% define custom key
% 	{unit}% key
% 	{\glsentrytext{\glslabel}}% default value
% 	{\glsentryunit}% command analogous to \glsentrytext
% 	{\GLsentryunit}% command analogous to \Glsentrytext
% 	{\glsunit}% command analogous to \glstext
% 	{\Glsunit}% command analogous to \Glstext
% 	{\GLSunit}% command analogous to \GLStext
% \glssetnoexpandfield{unit}
% \makeglossaries % create makeindex files

% \newglossarystyle{symbolsliststyle}{%
% 	\setglossarystyle{long3col}% style based on long3col
% 	\renewenvironment{theglossary}{%
% 		\begin{longtable}{lp{\glsdescwidth}>{\arraybackslash}p{2cm}}}%
% 		{\end{longtable}}%
% 	\renewcommand*{\glossaryheader}{% change the table header
% 		\bfseries Symbol & \bfseries Description & \bfseries Unit\\\hline%
% 		\endhead}%
% 	\renewcommand*{\glossentry}[2]{% change the displayed items
% 		\glstarget{##1}{\glossentryname{##1}}% name
% 		& \glossentrydesc{##1}% description
% 		& $\glsentryunit{##1}$% unit
% 		\tabularnewline
% 	}%
% }

% ****************************************************************************************
% Page layout and headers
% ****************************************************************************************
% Specify page layout (paper name and orientation specified in document class options)
\usepackage[
	includeheadfoot, % includes the head of the page into total body
	ignoremp, % disregards marginal notes in determining the horizontal margins
	nomarginpar, % shrinks spaces for marginal notes to 0pt
	hmargin=1.5in, % left and right margin
	vmargin=1in, % top and bottom margin
	headheight=14pt %  height of header
]{geometry}

\usepackage{parskip} % helps in implementing paragraph layouts

% Header and footer settings
\usepackage{fancyhdr}
\pagestyle{fancy} % set page style to 'fancy'
\renewcommand{\chaptermark}[1]{\markboth{\thechapter.\ #1}{}}
% \renewcommand{\sectionmark}[1]{\markright{\thesection.\ #1}}
\fancyhf{} % clear all header and footer fields
\fancyhead[L]{\leftmark} % set left header location (chapter)
% \fancyhead[R]{\rightmark} % set right header location (section)
\fancyfoot[C]{\thepage} % set center footer location (page count)

% ****************************************************************************************
% Packages for testing purposes (can be deleted)
% ****************************************************************************************
\usepackage{lipsum}
\usepackage{blindtext}
\usepackage{todonotes}
% EOF